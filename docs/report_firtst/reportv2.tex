\documentclass[UTF8,a4paper,10pt]{ctexart}
\usepackage[left=2.50cm, right=2.50cm, top=2.50cm, bottom=2.50cm]{geometry} %页边距
\CTEXsetup[format={\Large\bfseries}]{section} %设置章标题居左
 
 
%%%%%%%%%%%%%%%%%%%%%%%
% -- text font --
% compile using Xelatex
%%%%%%%%%%%%%%%%%%%%%%%
% -- 中文字体 --
%\setmainfont{Microsoft YaHei}  % 微软雅黑
%\setmainfont{YouYuan}  % 幼圆    
%\setmainfont{NSimSun}  % 新宋体
%\setmainfont{KaiTi}    % 楷体
%\setCJKmainfont{AR PL SungtiL GB}   % 宋体
%\setmainfont{SimHei}   % 黑体
% -- 英文字体 --
%\usepackage{times}
%\usepackage{mathpazo}
%\usepackage{fourier}
%\usepackage{charter}
\usepackage{helvet}
 
\usepackage[colorlinks,linkcolor=black, anchorcolor=green, citecolor=black]{hyperref} 
\usepackage{amsmath, amsfonts, amssymb} % math equations, symbols
\usepackage[english]{babel}
\usepackage{color}      % color content
\usepackage{graphicx}   % import figures
%\usepackage[colorlinks, linkcolor=blue]{url}        % hyperlinks
\usepackage{bm}         % bold type for equations
\usepackage{multirow}
\usepackage{booktabs}
\usepackage{epstopdf}
\usepackage{epsfig}
\usepackage{algorithm}
\usepackage{algorithmic}
\usepackage{tcolorbox}
\usepackage{colortbl}
%\usepackage{float}
\renewcommand{\algorithmicrequire}{ \textbf{Input:}}     % use Input in the format of Algorithm  
\renewcommand{\algorithmicensure}{ \textbf{Initialize:}} % use Initialize in the format of Algorithm  
\renewcommand{\algorithmicreturn}{ \textbf{Output:}}     % use Output in the format of Algorithm   
 



\usepackage{fancyhdr} %设置页眉、页脚
%\pagestyle{fancy}
\lhead{}
\chead{}
%\rhead{\includegraphics[width=1.2cm]{fig/ZJU_BLUE.eps}}
\lfoot{}
\cfoot{}
\rfoot{}
 
%%%%%%%%%%%%%%%%%%%%%%
% 中文映射
%%%%%%%%%%%%%%%%%%%%%



%%%%%%%%%%%%%%%%%%%%%%%
%  设置水印
%%%%%%%%%%%%%%%%%%%%%%%
%\usepackage{draftwatermark}         % 所有页加水印
%\usepackage[firstpage]{draftwatermark} % 只有第一页加水印
% \SetWatermarkText{Water-Mark}           % 设置水印内容
% \SetWatermarkText{\includegraphics{fig/ZJDX-WaterMark.eps}}         % 设置水印logo
% \SetWatermarkLightness{0.9}             % 设置水印透明度 0-1
% \SetWatermarkScale{1}                   % 设置水印大小 0-1    
 
\usepackage{hyperref} %bookmarks
\hypersetup{colorlinks, bookmarks, unicode} %unicode
 
 
 
\title{\textbf{机器翻译开题报告}}
\author{贾栋\; 王坤}
\date{\today}
 
\begin{document}
    \maketitle


\begingroup % start a TeX group
\color{black}% or whatever color you wish to use
\renewcommand{\contentsname}{目录}
\tableofcontents
\newpage
\endgroup   % end of TeX group
 

%\renewcommand{\abstractname}{抽象}
%\begin{abstract}
%这是一篇中文小论文。这个部分用来写摘要。摘要的章标题默认是英文,还没找到改成中文的方法:(
%\end{abstract} 

\section{实验目标}


%\subsection{实验目标}
此次实验的目标是基于现有研究成果,实现一个中英互译的机器翻译引擎,
可以提供中英互译的基本API。利用实验提供的UMCORPUS 数据集以及其他的一些开源数据集(详情见下文)
进行训练。根据 \href{http://nlpprogress.com/english/machine_translation.html}{\color{blue}nlpprocess} 的记录,
目前在机器翻译领域(基于WMT2014数据集)表现最为优秀的模型是 \href{https://arxiv.org/pdf/1808.09381.pdf}{\color{blue}Transformer Big + BT (Edunov et al., 2018)} (英德互译),
\href{https://www.deepl.com/press.html}{\color{blue}DeepL}(英法互译),以及表现同样较为优秀的 \href{https://arxiv.org/abs/1901.10430.pdf}{\color{blue}DynamicConv (Wu et al, 2019)}
等。其中DeepL是商业软件,
  其他两个都有论文发表。实验的基本思路是借鉴这些优秀论文的思想,比较具体的可行性,
  选取其中的一个进行实现。

从机器翻译的基本方法出发,本次实验选取了能够代表三大方法的开源 ML 引擎进行对比。包含但不限于以下模型:
\begin{itemize}

\item[$$$\bullet$] \href{http://www.statmt.org/moses/?n=Moses.Overview}{\color{blue}\underline{Mose}}  基于统计的 

\item[$\bullet$] \href{https://github.com/apertium}{\color{blue}\underline{Apertium}}                基于规则的

\item[$\bullet$] \href{http://thumt.thunlp.org/}{\color{blue}\underline{THUMT}}                     基于神经网络的
\end{itemize}
   
  最终选用\href{https://arxiv.org/pdf/1808.09381.pdf}{\color{blue}Transformer Big + BT (Edunov et al., 2018)} 作为最终实现目标。
  首先需要建立 Transformer 的基本模型,然后此基础上再进行改进。


\section{实验思路}

   从 MT 基于规则的方法,到依赖于大量数据的基于统计的方法,再到基于深度学习的 seq2seq learning。
   机器翻译领域不断涌现着新的思路和方法,并且表现出更优异的性能。从seq2seq的思路出发,机器翻译模型
   通过将一种语言的输入序列转化为另一种语言的输出序列,从而达到翻译的目的。基本的模型由两个 RNN 组成,
   一个encoder,一个decoder。
   

   由于RNN具有保存先前状态的能力,所以可以学习序列化数据的规律。之后在seq2seq的方法的基础之上,提出了attention机制,
   使得模型性能具有了较大提升。之后又提出的 \href{https://arxiv.org/pdf/1808.09381.pdf}{\color{blue}Transformer Big + BT (Edunov et al., 2018)} 
   不再使用RNN作为encoder$以及$decoder,而是全部使用attention。建立了直接的长距离依赖。


   所以现有的基本思路是建立 Transformer 的模型,通过研读有关论文,在此基础上使用 Back-Translation 的方法。
   最终目标是最大近似的再现\href{https://arxiv.org/pdf/1808.09381.pdf}{\color{blue}Transformer Big + BT}的模型。

\end{document}
